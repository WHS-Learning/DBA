% --- chapters/chap_algebra.tex ---
% Dieses Dokument enthält nur den Inhalt des Kapitels "Die Relationale Algebra"

\chapter{Die Relationale Algebra}

\section{Die Relationale Algebra}

Eine Algebra wird über einer Menge von Operanden und einer Operatorenmenge definiert.

Die Operationen sind dabei abgeschlossen über der gegebenen Menge,
d.h. die Anwendung eines Operators auf einem Element der Menge
resultiert wieder in einem Element der Menge.

\section{Operationen in der Relationalen Algebra}

Die relationale Algebra basiert auf der Mengenlehre.
Beispiele für Vereinigung ($\cup$), Schnitt ($\cap$) und Mengendifferenz ($\setminus$):
Seien $A = \{(1,1), (1,3), (2,5)\}$ und $B = \{(1,1), (1,2), (3,1)\}$.
\begin{itemize}
    \item $A \cup B = \{(1,1), (1,2), (1,3), (2,5), (3,1) \}$
    \item $A \cap B = \{(1,1)\}$
    \item $A \setminus B = \{(1,3), (2,5)\}$
\end{itemize}

Für die Anwendung in der Relationalen Algebra müssen die Relationen (Tabellen) dasselbe Schema (Spaltennamen und -typen) besitzen.

\par\medskip\noindent % \par\medskip erzeugt einen kleinen Abstand
Nehmen wir an:
% --- Hier beginnen die 'center' und 'tabular' Umgebungen statt ASCII-Art ---
\begin{center}
    \begin{minipage}[t]{0.45\textwidth}
        \centering
        \textbf{Tabelle A:}
        \begin{tabular}{ll}
            \toprule
            Film-ID & Titel \\
            \midrule
            1 & Wonka \\
            2 & Dune 2 \\
            4 & Oppenheimer \\
            \bottomrule
        \end{tabular}
    \end{minipage}
    \hfill
    \begin{minipage}[t]{0.45\textwidth}
        \centering
        \textbf{Tabelle B:}
        \begin{tabular}{ll}
            \toprule
            Film-ID & Titel \\
            \midrule
            1 & Wonka \\
            3 & Barbie \\
            5 & John Wick 4 \\
            \bottomrule
        \end{tabular}
    \end{minipage}
\end{center}

\begin{center}
    \textbf{Tabelle C:}
    \begin{tabular}{ll}
        \toprule
        Serien-ID & Titel \\
        \midrule
        1 & Euphoria \\
        2 & Sherlock \\
        \bottomrule
    \end{tabular}
\end{center}

Da die Schemata von A \& B identisch sind, können die
Mengenoperationen auf diese Mengen angewendet werden.
Das Schema von C unterscheidet sich von den anderen beiden, daher
sind Mengenoperationen wie bspw. $A \cup C$ \textbf{undefiniert} (und NICHT die leere Menge).

\par\medskip\noindent
\textbf{Beispiel $A \cup B$:} (Für $A \cup B$ müssen A und B nicht disjunkt sein.)
\begin{center}
    \textbf{Tabelle $A \cup B$:}
    \begin{tabular}{ll}
        \toprule
        Film-ID & Titel \\
        \midrule
        1 & Wonka \\
        2 & Dune 2 \\
        3 & Barbie \\
        4 & Oppenheimer \\
        5 & John Wick 4 \\
        \bottomrule
    \end{tabular}
\end{center}

\par\medskip\noindent
\textbf{Beispiel $A \setminus B$:} (Für $A \setminus B$ muss B keine Teilmenge von A sein.)
\begin{center}
    \textbf{Tabelle $A \setminus B$:}
    \begin{tabular}{ll}
        \toprule
        Film-ID & Titel \\
        \midrule
        2 & Dune 2 \\
        4 & Oppenheimer \\
        \bottomrule
    \end{tabular}
\end{center}

\par\medskip\noindent
\textbf{Beispiel $A \cap B$:}
\begin{center}
    \textbf{Tabelle $A \cap B$:}
    \begin{tabular}{ll}
        \toprule
        Film-ID & Titel \\
        \midrule
        1 & Wonka \\
        \bottomrule
    \end{tabular}
\end{center}

\subsection{Selektion ($\sigma$)}

Sei R eine Relation und $\phi$ (phi) eine beliebige Bedingung. Die Selektion $\sigma_{\phi}(R)$
bildet R auf eine Teilmenge $R'$ ab, in der alle Tupel die Bedingung $\phi$
erfüllen.

\par\medskip\noindent
Beispiel: Sei $R = \{(1, 1), (1, 2), (2, 1), (2, 2), (2, 3)\}$ und $sch(R) = (A, B)$.
Dann ist $\sigma_{A>1}(R) = \{(2, 1), (2, 2), (2, 3)\}$.

% --- Visuelle Darstellung der Operation (benötigt 'array' und 'amsmath' Pakete) ---
\[ % Beginnt eine zentrierte, mathematische Umgebung
\sigma_{\text{Dauer} < 120} \left(
    \parbox{4cm}{\centering % Eine Box für die linke Tabelle
    \begin{tabular}{lr}
        \toprule
        Titel & Dauer \\
        \midrule
        Wonka & 117 \\
        Dune 2 & 166 \\
        Barbie & 114 \\
        Oppenheimer & 180 \\
        John Wick 4 & 169 \\
        \bottomrule
    \end{tabular}}
\right)
=
\parbox{4cm}{\centering % Eine Box für die rechte Tabelle
\begin{tabular}{lr}
    \toprule
    Titel & Dauer \\
    \midrule
    Wonka & 117 \\
    Barbie & 114 \\
    \bottomrule
\end{tabular}}
\] % Ende der mathematischen Umgebung

Die Bedingung $\phi$ ist ein boolescher Ausdruck aus folgenden Operanden und Operatoren:
\begin{itemize}
    \item Konstanten
    \item Attribute
    \item arithmetische Operatoren ($+$, $-$, $*$, \dots)
    \item Vergleiche ($=$, $>$, $<$, \dots)
    \item Boolesche Operatoren ($\vee$ (oder), $\wedge$ (und), $\neg$ (nicht))
\end{itemize}
$\phi$ wird für jedes Tupel einzeln ausgewertet.
Es sind keine Quantoren ($\exists$, $\forall$) oder verschachtelte Algebraausdrücke
erlaubt.

\subsection{Projektion ($\pi$)}

Sei R eine Relation und L eine Attributsliste. Die Projektion $\pi_{L}(R)$ bildet
R auf eine Relation $R'$ ab, in der alle Tupel aus R enthalten sind, bei denen die
Attribute, die nicht in L sind, jeweils entfernt wurden.

\par\medskip\noindent
Beispiel:
Sei $R = \{(1, 1, 1), (1, 2, 3), (2, 2, 2), (3, 3, 3)\}$ und $sch(R) = (A, B, C)$.
Dann ist $\pi_{A,C} (R) = \{(1, 1), (1, 3), (2, 2), (3, 3)\}$.
\begin{itemize}
    \item $\pi$ kann auch dazu genutzt werden, die Spalten einer Tabelle neu zu sortieren.
    \item Intuitiv: $\sigma$ entfernt Zeilen, $\pi$ entfernt Spalten.
\end{itemize}
Durch die Projektion kann sich die Anzahl der Zeilen in der Tabelle ändern (Duplikate werden entfernt).

\subsection{Das kartesische Produkt ($\times$)}

Das kartesische Produkt $R \times S$ kombiniert jedes Tupel aus R mit jedem Tupel aus S.

\par\medskip\noindent
\textbf{Beispiel:}
\begin{center}
    \begin{minipage}[t]{0.3\textwidth}
        \centering
        \textbf{Relation R:}
        \begin{tabular}{cc}
            \toprule A & B \\ \midrule 1 & 1 \\ \bottomrule
        \end{tabular}
    \end{minipage}
    \hfill
    \begin{minipage}[t]{0.3\textwidth}
        \centering
        \textbf{Relation S:}
        \begin{tabular}{cc}
            \toprule C & D \\ \midrule 4 & 4 \\ 4 & 5 \\ \bottomrule
        \end{tabular}
    \end{minipage}
    \hfill
    \begin{minipage}[t]{0.3\textwidth}
        \centering
        \textbf{Relation T:}
        \begin{tabular}{c}
            \toprule E \\ \midrule 8 \\ 9 \\ \bottomrule
        \end{tabular}
    \end{minipage}
\end{center}

\begin{center}
    \textbf{Relation $R \times S$:}
    \begin{tabular}{cccc}
        \toprule
        A & B & C & D \\
        \midrule
        1 & 1 & 4 & 4 \\
        1 & 1 & 4 & 5 \\
        \bottomrule
    \end{tabular}
\end{center}

\begin{center}
    \textbf{Relation $(R \times S) \times T$:}
    \begin{tabular}{ccccc}
        \toprule
        A & B & C & D & E \\
        \midrule
        1 & 1 & 4 & 4 & 8 \\
        1 & 1 & 4 & 5 & 8 \\
        1 & 1 & 4 & 4 & 9 \\
        1 & 1 & 4 & 5 & 9 \\
        \bottomrule
    \end{tabular}
\end{center}

\subsection{Umbenennung ($\beta$)}

Seien A und B Attribute und R eine Relation. Die Umbenennung $\beta_{B \leftarrow A}(R)$
(beta) benennt die Spalte A in R in B um.

% --- Visuelle Darstellung der Operation ---
\[
\beta_{\text{ID} \leftarrow \text{Film-ID}} \left(
    \parbox{5cm}{\centering
    \begin{tabular}{ll}
        \toprule
        Film-ID & Titel \\
        \midrule
        1 & Wonka \\
        2 & Dune 2 \\
        3 & Barbie \\
        4 & Oppenheimer \\
        5 & John Wick 4 \\
        \bottomrule
    \end{tabular}}
\right)
=
\parbox{5cm}{\centering
\begin{tabular}{ll}
    \toprule
    ID & Titel \\
    \midrule
    1 & Wonka \\
    2 & Dune 2 \\
    3 & Barbie \\
    4 & Oppenheimer \\
    5 & John Wick 4 \\
    \bottomrule
\end{tabular}}
\]

\section{Rechengesetze und Eigenschaften}

Das kartesische Produkt ist assoziativ:
\[ R \times (S \times T) \equiv (R \times S) \times T \]

Das kartesische Produkt ist nicht kommutativ. (Außer in Kombination mit
einer Projektion, die die Spalten neu sortiert):
\[ \pi_{L}(R \times S) \equiv \pi_{L}(S \times R) \]

Außerdem gilt: Gegeben $\sigma_{\phi}$ enthält nur Attribute von R, dann:
\[ \sigma_{\phi}(R \times S) \equiv \sigma_{\phi}(R) \times S \]

\section{Join-Operationen}
% TODO: Inhalt
\section{Anfragebäume und Anfrageoptimierung (kurz)}
% TODO: Inhalt